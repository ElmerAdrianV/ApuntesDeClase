\documentclass[12pt]{article}
\usepackage[spanish, mexico]{babel}
\usepackage{amsmath}
\usepackage{amsthm}
\usepackage{amssymb}
\usepackage{amsfonts}
\usepackage{IEEEtrantools}
\usepackage{caption}
\usepackage{listings}
\usepackage{graphicx}
\graphicspath{ {images/} }
\pagestyle{headings}
\begin{document}
\title{Apuntes sobre Cálculo Diferencial e Integral III}
\author{Elmer Ortega}
\maketitle
\noindent\section*{Un pequeño recordatorio sobre vectores}
Se conoce a un \textbf{vector} como la conexión de $n$ números 
reales. Se denota como $\vec{v}\in\mathbb{R}^n, \vec{v}=\left(v_1,v_2,v_3\dots,v_n\right)$. 
Además, el conjunto $\mathbb{R}^n$ se define como $\mathbb{R}^n=\left\{a_1,a_2,a_3\dots,a_n | a_i\in\mathbb{R}\right\}$
\newline
\newline
\textbf{Definición de la suma}\newline
Sea $\vec{v}=\left(v_1,v_2,v_3\dots,v_n\right), \vec{u}=\left(u_1,u_2,u_3\dots,u_n\right)$, se define la suma de $\vec{v}+\vec{u}$ como:
\begin{center}
    $\vec{v}+\vec{u}=\left(v_1+u_1,v_2+u_2,\dots,v_i+u_i,\dots,v_n+u_n\right)$
\end{center}
\textbf{Definición del producto por un escalar}\newline
Sea $\lambda\in\mathbb{R},\vec{v}=\left(v_1,v_2,v_3\dots,v_n\right)$, se define el producto de $\lambda*\vec{v}$ como
\begin{center}
    $\lambda*\vec{v}=\left(\lambda v_1, \lambda v_2+,\dots,\lambda v_i,\dots,\lambda v_n\right)$
\end{center}
Todos los $\mathbb{R}^n$ con estas definiciones son un \textbf{espacio vectorial}.
\newline
\textbf{La base canónica en $\mathbb{R}^n$}\newline
Se define a la base canónica de $\mathbb{R}^n$ como el conjunto de:
\begin{center}
    $\vec{e}_1=\left(1,0,0\dots,0\right), \vec{e}_2=\left(0,1,0\dots,0\right), \dots, \vec{e}_n=\left(0,0,0\dots,1\right)$
\end{center}
En especial, la base canónica de $\mathbb{R}^2$ se define como $\left\{i=(1,0),j=(0,1)\right\}$ y $\mathbb{R}^3=\{i=(1,0,0), j=(0,1,0),k=(0,0,1)\}$
Y, ¿por qué es útil? porque puedes pensar los vectores como: 
$\left(a_1,a_2,a_3\dots,a_n\right)=a_1\vec{e}_1+a_2\vec{e}_2+\dots+a_n\vec{e}_n$ 
que es un vector como la combinación lineal de la base canónica y es útil a la hora de diferenciar e integrar.
\newpage
\noindent\textbf{Definición de norma} \newline
Sea el vector $\vec{v}\in\mathbb{R}^n, \vec{v}=(v_1,v_2,\dots,v_n)$, se define la norma de $\vec{v}$ como:\newline
$\|\vec{v}\|=\sqrt{v_1^{2}+v_2^{2}+v_3^{2}+\dots+v_n^2}$, que es igual a la 
\emph{distancia del punto al origen.} \newline
\newline
\textbf{Propiedades de la norma}
\begin{enumerate}
     \item $\|\vec{v}\|\geq0$
     \item $\|\vec{v}\|=0 \Leftrightarrow \vec{v}=\vec{0}$
     \item $\|\lambda\vec{v}\|=|\lambda|\|\vec{v}\|$
     \item Desigualdad del triángulo: $\|\vec{u}+\vec{v}\|\leq \|\vec{u}\|+\|\vec{v}\|$, la igualdad se da 
     $\Leftrightarrow\vec{u}=\lambda\vec{v},\text{ con }\lambda\geq0$
\end{enumerate}

\noindent\textbf{Definición de producto punto o producto interno} \newline
Para  $\vec{u},\vec{v}\in\mathbb{R}^n$, se define el producto punto de $\vec{u}\cdot\vec{v}$ como:
\begin{center}
    $\vec{u}\cdot\vec{v}=u_1v_1+u_2v_2+\dots+u_nv_n$     
\end{center}
\textbf{Propiedades del producto punto}
\begin{enumerate}
     \item $\left(\vec{u}+\vec{v}\right)\cdot\vec{w} = \vec{u}\cdot\vec{w}+\vec{v}\cdot\vec{w}$
     \item $\left(\lambda\vec{u}\right)\cdot\vec{v}=\lambda\left(\vec{u}\cdot\vec{v}\right)$
     \item $\vec{u}\cot\vec{v}=\vec{v}\cot\vec{u}$
     \item $\vec{u}\cdot\vec{u}\geq0$
     \item $\vec{u}\cdot\vec{u}=0\Leftrightarrow\vec{u}=\vec{0}$
     \item $\sqrt{\vec{u}\cdot\vec{u}} = \|\vec{u}\|$
     \item $\cos{\theta}=\dfrac{\vec{u}\cdot\vec{v}}{\|\vec{u}\|}\|\vec{v}\|$, donde $\theta$ es el ángulo
     entre $\vec{u}$ y $\vec{v}$
     \newline
     \textbf{Nota}: $\vec{u}\bot\vec{v}$ (son ortogonales) 
     $\Leftrightarrow$ $\theta=90°=\dfrac{\pi}{2}^r\Leftrightarrow\vec{u}\cdot\vec{v}=0$
\end{enumerate}
\newpage
\noindent\textbf{Representación paramétrica de la recta}
\begin{center}
    \includegraphics{Imagen1.png}
\end{center}
En general, dado $\vec{p}\in\mathbb{R}^n,\vec{d}\in\mathbb{R}^n, \vec{d}\not=\vec{0}, t\in\mathbb{R}$,
\begin{center}
    $\vec{v}=\vec{p}+t\vec{d}$ representa una línea recta en $\mathbb{R}^n$
\end{center}
\textbf{Representación de un hiper-plano}\newline
En general, en $\mathbb{R}^n$, la ecuacion del hiper-plano está dada por,
\begin{center}
    $\left(\vec{v}-\vec{p}\right)\cdot\vec{n}=0$, que es un objeto de dimensión $\mathbb{R}^{n-1}$
\end{center}
Por ejemplo, en $\mathbb{R}^3$, con $\vec{n}=(n_1,n_2,n_3), \vec{p}=(p_1,p_2,p_3), \vec{v}=(x,y,z)$, se tiene que $n_1x+n_2y+n_3z-(n_1p_1+n_2p_2+n_3p_3)=0$, 
por lo que, dado una ecuacion $Ax+By+Cz+F=0$, el vector $\vec{d}=(A,B,C)$ es perpendicular a los vectores $\vec{v}, \vec{p}$
\begin{center}
    \includegraphics{Imagen1.png}
\end{center}
Dado, $\vec{n}$ perpendicular a la recta $\vec{p}+t\vec{d}$, con $\vec{d}=\vec{v}-\vec{p} \Rightarrow \left(\vec{v}-\vec{p}\right)\cdot\vec{n}=0$

\newpage
\section{Funciones y diferenciación}
\noindent$ \bullet$ Se conoce como funciones con valores escalares o campos escalares a las funciones del tipo $f: \mathbb{R}^n\rightarrow\mathbb{R}$\newline
Por ejemplo, $f: \mathbb{R}^2\rightarrow\mathbb{R}, f(x,y)=3x^2+5xy+e^y$\newline
\newline
$ \bullet$ Se conoce como funciones con valores escalares o campos escalares a las funciones del tipo $f: \mathbb{R}^n\rightarrow\mathbb{R}$\newline
Por ejemplo, $f: \mathbb{R}^2\rightarrow\mathbb{R}, f(x,y)=3x^2+5xy+e^y$\newline
\newline
\textbf{Definición:} Sea $f: \mathbb{R}^n\rightarrow\mathbb{R}$ la \textbf{gráfica} de $f$ es
\begin{center}
    $\left\{(x_1,x_2,x_3,\dots,x_n,f(x_1,x_2,x_3,\dots,x_n)):x_1,x_2,x_3,\dots,x_n\in\mathbb{R}\right\}$
\end{center}
\textbf{Definición:} Se define a las \textbf{curvas de nivel} de $f:\mathbb{R}^2\rightarrow\mathbb{R}$ como
\begin{center}
    $L_c:=\left\{(x,y)\in\mathbb{R}^2: f(x,y)=c\right\}$
\end{center}
\textbf{Definición:} Se define a las \textbf{superficies de nivel} de  $f: \mathbb{R}^3\rightarrow\mathbb{R}$ como
\begin{center}
    $L_c:=\left\{(x,y,z)\in\mathbb{R}^3: f(x,y,z)=c\right\}$
\end{center}
*Notas: La gráfica $f:\mathbb{R}^3\rightarrow\mathbb{R}$ es un objeto que vive en $\mathbb{R}^4$. En general,
 la gráfica $f:\mathbb{R}^n\rightarrow\mathbb{R}$ es un objeto que vive en $\mathbb{R}^{n+1}$.\newline
\textbf{Definición:} Se define al \textbf{conjunto de nivel} de  $f: \mathbb{R}^n\rightarrow\mathbb{R}$ como
 \begin{center}
     $L_c:=\left\{(x_1,x_2,\dots,x_n)\in\mathbb{R}^n: f(x_1,x_2,x_3,\dots,x_n)=c\right\}$
 \end{center}
\subsection*{Conjuntos abiertos y conjuntos cerrados}
\noindent\textbf{Definición:} Dado $\vec{x}_0\in\mathbb{R}^n$ y $\epsilon>0$, la \textbf{bola (abierta)} es el conjunto
 \begin{center}
     $B_{\epsilon}(\vec{x}_0)=\left\{\vec{x}\in\mathbb{R}^n:\|\vec{x}-\vec{x}_0\|<\epsilon\right\}$\newline
     Se le denota como la bola de radio $\epsilon$ centrada en $\vec{x}_0$
 \end{center}
 \textbf{Definición:} Dado un conjunto $A\subseteq\mathbb{R}^n$ es \textbf{abierto} si 
 \begin{center}
     $\forall \vec{x}_0\in A$,existe un $\epsilon > 0 $ tal que $B_\epsilon(\vec{x}_0)\subseteq A$ 
 \end{center}
 \textbf{Proposición:} La bola $B_\epsilon(\vec{x}_0)$ es abierta.\newline
 \textbf{Pd.} $\forall B_\epsilon(\vec{y}_0)\subseteq B_\epsilon(\vec{x}_0)$. Sea $\vec{z}_0\in B_{\epsilon}(\vec{y}_0)$
 \begin{center}
    $\|\vec{z}_0-\vec{x}_0\| = \|\vec{z}_0+\vec{y}_0-\vec{y}_0-\vec{x}_0\|\leq \|\vec{z}_0-\vec{y}_0\|+\|\vec{y}_0-\vec{x}_0\|$ \newline
    $\Rightarrow \|\vec{z}_0-\vec{y}_0\|+\|\vec{z}_0-\vec{x}_0\| < \epsilon - \|\vec{y}_0-\vec{x}_0\|=\epsilon$\newline
    $\Rightarrow \vec{z}_0\in B_{\epsilon}(\vec{x}_0) \therefore \text{ la bola }  B_\epsilon(\vec{x}_0)$ es abierta.
 \end{center}
 \newpage
 \noindent \textbf{Propiedades de los conjuntos abiertos}
\begin{enumerate}
    \item Si $V_1,V_2,V_3,\dots, V_n$  son conjuntos abiertos en $\mathbb{R}^n$, entonces $V_1\cup V_2\cup V_3\cup\dots V_n$ es abierto (Aplica para una union finita e infinita).
    \item Si $V_1,V_2,V_3,\dots, V_n$  son conjuntos abiertos en $\mathbb{R}^n$, entonces $V_1\cap V_2\cap V_3 \cap \dots V_n$ es abierto (Solo para una intersección finita).
    \item $\mathbb{R}^n$ es abierto.
    \item $\emptyset$ es abierto. $\rightarrow$ argumento de vacuidad o vacío.
\end{enumerate}
\noindent\subsection*{Puntos frontera}
\noindent \textbf{Definición: } Sea $A \in \mathbb{R}_n$. Un punto $\vec{x}_0 \in \mathbb{R}^n$ es un \textbf{punto frontera}
de $A$ si todas las bolas abiertas centradas en $\vec{x}_0$ tienen en $A$ y puntos no en $A$. \newline \newline
El conjunto de puntos frontera de $A$ se llama la \textbf{frontera de } $A$.\newline
\textbf{Ejemplos de puntos fronteras:}

\begin{center}
    \begin{tabular}{| c | c |}
    \hline
    Conjunto $A$ & frontera de $A$ \\
    \hline
    $A=(-2,3)\subseteq\mathbb{R}$ & $\{-2,3\}$  \\
    $A=[-2,3)\subseteq\mathbb{R}$ & $\{-2,3\}$  \\
    $A=(-2,3)\cup\{ 6 \}\subseteq\mathbb{R}$ & $\{-2,3,6\}$  \\
    $A=(-2,3)\cup[3,6)\subseteq\mathbb{R}$ & $\{-2,6\}$  \\
    $A=(-2,3)\cup(3,6]\subseteq\mathbb{R}$ & $\{-2,3,6\}$  \\
    $A=\{x\in\mathbb{Q} : 0\geq x \geq 1\}$ & $[0,1]$ \\
    $A=\{(x,y)\in\mathbb{R}^2:x^2+y^2<1\}$ & $\{(x,y)\in\mathbb{R}^2:x^2+y^2=1\}$ \\
    $A$ es un finito& El mismo conjunto $A$\\
    $A=\left\{n+\dfrac{1}{n}:n\in\mathbb{N}\right\}=\left\{2,1+\dfrac{1}{2},\dots\right\}$ & $A\cup \{1\}$\\
    $A=\mathbb{R}^n$ & $\emptyset$\\
    $A=\emptyset$ & $\emptyset$\\
    \hline
    \end{tabular}
    \end{center}
\newpage
\noindent \textbf{Definición: }  Sea $B\subseteq\mathbb{R}^n$. Decimos que $B$ es un conjunto \textbf{cerrado} si $B^c$ es \emph{abierto}.
\newline\newline
\textbf{Propiedades de conjuntos cerrados: }
\begin{enumerate}
    \item Si $V_1,V_2,V_3,\dots, V_n$  son conjuntos cerrados en $\mathbb{R}^n$, entonces $V_1\cup V_2\cup V_3\cup\dots V_n$ es abierto (Solo para una intersección finita).
    \item Si $V_1,V_2,V_3,\dots, V_n$  son conjuntos abiertos en $\mathbb{R}^n$, entonces $V_1\cap V_2\cap V_3 \cap \dots V_n$ es abierto (Aplica para una union finita e infinita).
    \item $\emptyset$ es cerrado.
    \item$\mathbb{R}^n$ es cerrado.
\end{enumerate}
\noindent\textbf{*Nota: }$\empty$ y $\mathbb{R}^n$ son los únicos conjuntos abiertos y cerrados.\newline
\noindent\textbf{*Nota: }La frontera de $A$ y $A^c$ es el mismo conjunto. \newline
\noindent\textbf{*Nota: }Si $A$ es un conjunto finito de $\mathbb{R}^n$, entoces $A$ es cerrado y no abierto.\newline
\noindent\textbf{*Nota:}\"{}Entre 2 números racionales, hay un irracional y hay un racional entre 2 números irracionales\"{}.\newline
\newpage
\section*{Límites}
\noindent\textbf{Definición de límite:} Sea $f:A\subseteq\mathbb{R}^n\rightarrow\mathbb{R}^m$ con $A$ abierto y 
sea $\vec{a}\in A$\newline
Decimos que
\begin{center}
    $\lim_{\vec{x}\rightarrow\vec{a}}{f(\vec{x})} = \vec{b}$ , $(\vec{b}\in\mathbb{R}^m)$ 
\end{center}
\noindent Si $\forall \epsilon>0, \exists \delta >0 $ tq si $\vec{x}\in A$ y $\|\vec{x}-\vec{a}\| < \delta \Rightarrow  \|f(\vec{x})-\vec{b}\| < \epsilon$ .
\end{document}
